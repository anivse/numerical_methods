\include{settings}

\begin{document}	% начало документа

% Титульная страница
\begin{titlepage}	% начало титульной страницы

	\begin{center}		% выравнивание по центру

		\large Санкт-Петербургский политехнический университет Петра Великого\\
		\large Физико-механический институт \\
		\large Высшая школа прикладной математики и вычислительной физики\\[3cm]
		% название института, затем отступ 6см
		\large Направление подготовки\\
		\large "01.03.02. Прикладная математика и информатика"\\[3cm]
		\huge Дисциплина "Численные методы"\\[0.5cm] % название работы, затем отступ 0,5см
		\large Отчет по лабораторной работе №4\\[0.1cm]
		\large "Решение алгебраической проблемы собственных значений итерационными методами. Степенной метод для поиска второго максимального по модулю собственного значения и соответствующего собственного вектора"\\[5cm]

	\end{center}


	\begin{flushright} % выравнивание по правому краю
		\begin{minipage}{0.25\textwidth} % врезка в половину ширины текста
			\begin{flushleft} % выровнять её содержимое по левому краю

				\large\textbf{Работу выполнил:}\\
				\large Иванова А.С.\\
				\large {Группа:} 5030102/00002\\
				
				\large \textbf{Преподаватель:}\\
				\large Курц В.В.

			\end{flushleft}
		\end{minipage}
	\end{flushright}
	
	\vfill % заполнить всё доступное ниже пространство

	\begin{center}
	\large Санкт-Петербург\\
	\large \the\year % вывести дату
	\end{center} % закончить выравнивание по центру

\end{titlepage} % конец титульной страницы

\vfill % заполнить всё доступное ниже пространство


\include{ToC}

\section{Формулировка задачи}

Необходимо решить задачу Коши для обыкновенного дифференциального уравнения 2-го порядка методом Адамса 2-го порядка

\begin{equation}
	\begin{cases}
	  y''=f(x,y,y') \\
	  y'(a)=y_{0}' \\
	  y(a) = y_{0} 	
	\end{cases}
\end{equation}

Исходная функция: 
\begin{equation}
	y''=\frac{-(2x+2)*y'*x+y*x+1}{x^{2}*(2x+1)} \\
\end{equation}

На отрезке [0.2;1] с начальными условиями

\begin{math}
	y'(0.2)=-25 ;
	y(0.2) = 5 
\end{math}

Известно точное решение:

\begin{math}
	y=\frac{1}{x}
\end{math}

Необходимо исследовать сходимость метода, влияние шага на точность вычислений и влияние ошибок в исходных данных на решение, т.е. устойчивость задачи, сравнить полученные результаты с методом Эйлера-Коши.

\section{Алгоритм метода и условия его применимости}

\subsection{Алгоритм метода}

Решение задачи Коши будем искать в виде значений сеточной функции, построенной на равномерной сетке на отрезке [a;b]. 

Сделаем замену: 
\begin{math}
	z(x)=y'(x); z^{'}(x)=y''(x)
\end{math}

Тогда получим систему дифференциальных уравнений первого порядка: 

\begin{equation}
	\begin{cases}
		y'(x)=z(x) \\
		z'(x)=f(x,y,z) \\
		y(a) = y_{0} \\
		z(a) = y'_{0}
	\end{cases}
\end{equation}

Схема предиктор-корректор (метод прогноза и коррекции) — семейство алгоритмов численного решения различных задач, которые состоят из двух шагов. На первом шаге (предиктор) вычисляется грубое приближение требуемой величины. На втором шаге при помощи иного метода приближение уточняется (корректируется).

К данному семейству относится метод Адамса 2-го порядка.

Для дифференциального уранвения 1-го порядка: 

N - Количество разбиений равномерной сетки

\begin{math}
	h = \frac{b-a}{N}; x_{i}=a+h*i; i=0,...,N
\end{math}

Проинтегрируем уравнение \begin{math}
  y'=f(x,y) 
\end{math} по \begin{math}
 [x_{k-1},x_{k}]
\end{math}

\begin{equation}
	y_{k}=y_{k-1}+\int\limits_{x_{k-1}}^{x_{k}}f(x,y(x))dx=y_{k-1}+\int\limits_{x_{k-1}}^{x_{k}}F(x)dx
\end{equation}

Идея: аппроксимируем F(x) интерполяционным полиномом в форме Лагранжа

\begin{equation}
	y_{k}=y_{k-1}+\int\limits_{x_{k-1}}^{x_{k}}L_{m}(x)dx+\int\limits_{x_{k-1}}^{x_{k}}R_{m}(x)dx
\end{equation}

 Второе слагаемое - остаточный член полинома Лагранжа, им можно пренебречь
 
 Замена переменной: \begin{math}
 	x=x_{0}+h*t
 \end{math}

Пусть r - шаговость метода. 

Если m=r-1, получаем явную (экстраполяционную формулу)

\begin{equation}
	y_{k}=y_{k-1}+h*\sum\limits_{j=0}^{r-1}\beta_{j}^{(r)}f_{k-r+j}
\end{equation}

где

\begin{math}
	\beta_{j}^{(r)}=\frac{(-1)^{r-1-j}}{(r-1-j)!j!}\int\limits_{r-1}^{r}\frac{t(t-1)...(t-r+1)}{t-j}dt
\end{math}

Если m=r, получаем неявную формулу

\begin{equation}
	y_{k}=y_{k-1}+h*\sum\limits_{j=0}^{r}\beta_{j}^{(r)}f_{k-r+j}
\end{equation}

где

\begin{math}
	\beta_{j}^{(r)}=\frac{(-1)^{r-j}}{(r-j)!j!}\int\limits_{r-1}^{r}\frac{t(t-1)...(t-r)}{t-j}dt
\end{math}

Явную формулу будем использовать в качестве предиктора, неявную - в качестве корректора для метода Адамса. 

Получим формулы:

\begin{equation}
	\begin{cases}
		\tilde y_{i}=y_{i-1}+\frac{h}{2}*(3*f(x_{i-1},y_{i-1})-f(x_{i-2},y_{i-2})) \\
		y_{i}=y_{i-1}+h*\frac{f(x_{i-1},y_{i-1})+f(x_{i},\tilde y_{i})}{2}
	\end{cases}
\end{equation}

В нашем случае для системы дифференциальных уравнений:

\begin{equation}
	\begin{cases}
		\tilde y_{i} = y_{i-1} + \frac{h}{2} * (3*z_{i-1}-z_{i-2}) \\
		\tilde z_{i}= z_{i-1}+\frac{h}{2}*(3*f(x_{i-1},y_{i-1},z_{i-1})-f(x_{i-2},y_{i-2},z_{i-2})) \\
		y_{i} = y_{i-1} + h * \frac{z_{i-1}+\tilde z_{i}}{2} \\
		z_{i} = z_{i-1} + h * \frac{f(x_{i-1},y_{i-1},z_{i-1})+f(x_{i},\tilde y_{i},\tilde z_{i}) }{2} 
	\end{cases}
\end{equation}

Для вычисления промежуточных значений с индексом i-1 можно использовать метод Эйлера-Коши
\subsection{Условия применимости}

\begin{itemize}
	\item Частная производная по х непрерывна и ограничена
	\item Выполянется условие Липшица по y
	\begin{equation}
		|f(x,y_{1})-f(x,y_{2})| \leq L|y_{1}-y_{2}|
	\end{equation}
	\item Существование непрерывных производных до 2-го порядка для применения правила Рунге.
\end{itemize}

\section{Предварительный анализ задачи}

Для оценки погрешности используется правило Рунге:

\begin{equation} 
	\frac{|S_{n,2N}(f)-S_{n,N}(f)|}{2^{m}-1} \leq \epsilon
\end{equation}

Для метода Адамса m=2

\section{Проверка условий применимости метода}

Исходная функция: 
\begin{equation}
	y''=\frac{-(2x+2)*y'*x+y*x+1}{x^{2}*(2x+1)} \\
\end{equation}

На отрезке [0.2;1]

Данная функция будет иметь разрывы производных по х в точках 0 и -0.5, которые не входят в заданный отрезок, следовательно метод Адамса можно использовать. Условие Липшица по у выполнено. 


\section{Тестовый пример с детальными расчетами для задачи малой размерности}

Решим дифференциальное уравнение 1-го порядка:

\begin{equation}
	y'=\frac{2xy+3}{x^{2}} \\
\end{equation}

На отрезке [1;2] с начальным условием y(1)=-1

Ответ для проверки:

\begin{math}
	y=-\frac{1}{x}
\end{math}

Возьмем N=1, тогда h=1 (при N=1 не получится использовать метод Адамса)

\begin{math}
      \tilde y_{1}=y_{0}+h*f(x_{0},y_{0})=-1+\frac{2*1*(-1)+3}{1}=-1+1=0   
\end{math}

\begin{math}
	y_{1}=y_{0}+h*\frac{f(x_{0},y_{0})+f(x_{1},\tilde y_{1})}{2}= -1+\frac{1+0.75}{2}=-1+0.875=-0.125
\end{math}

Погрешность существенна. Пусть N=2, тогда h=0.5

\begin{math}
	\tilde y_{1}=y_{0}+h*f(x_{0},y_{0})=-1+0.5\frac{2*1*(-1)+3}{1}=-1+0.5=-0.5   
\end{math}

\begin{math}
	y_{1}=y_{0}+h*\frac{f(x_{0},y_{0})+f(x_{1},\tilde y_{1})}{2}=-0.5833
\end{math}

\begin{math}
	\tilde y_{2}=y_{1}+\frac{h}{2}*(3*f(x_{1},y_{1})-f(x_{0},y_{0}))=-0.4166   
\end{math}

\begin{math}
	y_{2}=y_{1}+\frac{h}{2} (f(x_{1},y_{1})+f(x_{2},\tilde y_{2}))=-0.3611
\end{math}
 
Дальнейшими итерациями можно приблизить значение к точному.
  
\section{Перечень контрольных тестов для иллюстрации метода}

 Необходимо решить задачу Коши для обыкновенного дифференциального уравнения 2-го порядка методом Адамса 2-го порядка
 
 \begin{equation}
 	\begin{cases}
 		y''=f(x,y,y') \\
 		y'(a)=y_{0}' \\
 		y(a) = y_{0} 	
 	\end{cases}
 \end{equation}
 
 Исходная функция: 
 \begin{equation}
 	y''=\frac{-(2x+2)*y'*x+y*x+1}{x^{2}*(2x+1)} \\
 \end{equation}
 
 На отрезке [0.2;1] с начальными условиями
 
 \begin{math}
 	y'(0.2)=-25 ;
 	y(0.2) = 5 
 \end{math}

Исследуется сходимость метода (количество итераций от заданной точности), влияние шага h на точность вычислений и влияние ошибок в исходных данных на решение, т.е. устойчивость задачи. Полученные результаты сравниваются с результатами для метода Эйлера-Коши 

\section{Модульная структура программы}

def my\_ddfunc(x, y, z):

Вычисление значения исходной функции от x,y,z

def answer(x):

Точное решение задачи Коши

def Adams(xmin, xmax, N, ddfunc, y0, z0):

Вычисление значения одной итерации метода Адамса при заданном количестве разбиений

def iterations(eps , xmin, xmax,ddfunc, y0, z0):

Получение решения задачи Коши с помощью метода Адамса с заданной точностью. 

\section{Численный анализ решения задачи}

\subsection{Сходимость метода}

\includegraphics[scale=0.75]{1.pdf}

Из данного графика можно сделать вывод, что для решения задачи Коши с заданной точностью для метода Адамса требуется меньше итераций, чем для метода Эйлера-Коши, что связано с тем, что метод Адамса является двухшаговым и в нем учитываются результаты на предыдущих узлах сетки. 

\subsection{Влияние шага на точность вычислений}

\includegraphics[scale=0.75]{2.pdf}

Из данного графика можно сделать вывод, что для метода Эйлера-Коши требуется меньший шаг равномерной сетки для достижения заданной точности. 

\subsection{Влияние ошибки исходных данных на решение}

\includegraphics[scale=0.75]{4.pdf}

Из данного графика можно сделать вывод, что при увеличении ошибки в исходных данных ошибка вычислений увеличивается. Данная задача является устойчивой, т.к. ошибка входных данных и ошибка результата имеют одинаковый порядок.

\section{Краткие выводы}

На основе полученных результатов можно сделать вывод о том, что при увеличении количества итераций и шага разбиения равнмоерной сетки погрешность результата уменьшается. Также можно сделать вывод о том, что если в исходные данные вносить ошибки, то при ее увеличении будет увеличиваться и ошибка результата. Данная задача является устойчивой, т.к. ошибка входных данных и ошибка результата имеют одинаковый порядок. 

Также можно сделать вывод, что для решения задачи Коши с заданной точностью для метода Адамса требуется меньше итераций, чем для метода Эйлера-Коши, что связано с тем, что метод Адамса является двухшаговым и в нем учитываются результаты на предыдущих узлах сетки.

\end{document}
